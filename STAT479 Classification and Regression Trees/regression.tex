 \documentclass[12pt]{article}
 %File creation date: 10/31/19  at 18:58
 \usepackage{pstricks,pst-node,pst-tree}
 \usepackage{geometry}
 \usepackage{lscape}
 \definecolor{wheat}{rgb}{0.96, 0.87, 0.7}
 \pagestyle{empty}
 \begin{document}
 %\begin{landscape}
 \begin{center}
\psset{linecolor=black,tnsep=1pt,tndepth=0cm,tnheight=0cm,treesep=.8cm,levelsep=50pt,radius=10pt}
  \pstree[treemode=D]{\TC~[tnpos=l]{\shortstack[r]{\texttt{\detokenize{PINCP}}\\$\leq_*$289500}}
 }{
    \TC[fillcolor=yellow,fillstyle=solid]~{\shortstack[c]{\emph{20057}\\459.5}}
    \TC[fillcolor=yellow,fillstyle=solid]~{\shortstack[c]{\emph{181}\\71208.}}
 }
 \end{center}
GUIDE v.32.0  0.20-SE
piecewise constant least-squares regression tree
for predicting \texttt{\detokenize{INTP}}.
 Number of observations used to contruct tree is 20238
 (excluding observations with non-positive weight or with missing values
 in d, t, r or z variables).
 Maximum number of split levels is 30 and minimum node sample size is 101.
At each split, an observation goes to the left branch 
 if and only if the condition is satisfied.
 The symbol `$\leq_*$' stands for `$\leq$ or missing'.
Sample size (\emph{in italics}) and mean of \texttt{\detokenize{INTP}} printed below nodes.
Second best split variable at root node is \texttt{\detokenize{POVPIP}}.
 %\end{landscape}
 \end{document}
