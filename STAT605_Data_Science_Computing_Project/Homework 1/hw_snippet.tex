 most typical formula for editing in vim is to combine an operator (e.g., {\tt d}, {\tt x} or {\tt y}) with a motion (i.e., specifying a location, such as  next beginning of a word or  end of  current line). We saw  {\tt dd} command in lecture, but  {\tt d} command has a lot more options available. For example, we can write {\tt dw} to delete from  cursor to  next start of a word, or {\tt d\^} to delete from  cursor to  beginning of  line. An especially useful command is  {\tt daw} command. It deletes  current word under  cursor (it's easy to remember-- {\tt daw} stands for ``delete a word''). Use  {\tt daw} command to delete  word {\tt approximately} from  last line of \verb|semicolons.c|.
